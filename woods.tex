Settings:
Glacier National Park
- 

Inciting Incidents
- Making bad decisions in pursuit of a romantic interest


What moves the plot forward



Core ideas/Themes - big quotes
People are resisitant to change




Scary Moments




Villains
- UFOs
- Hybrid animals
- Zombies
- Cryptids
- Mind controlled people
- Ghosts of native tribes
- Secret army of robots in the forest
- AI research labratory with rogue assets, could be robots like EMMI, or humans like Westworld, or animals


What about mixing these?

How would I subvert some sort of alien stereotype?
- also involves

What if the AI is 




Q Thoughts:
Definitely want pre-configured ideation axes because there

An important part of developing the system is developing good nomenclature to go with it. The whole "axes" nomenclature wasn't quite good enough because it was too technical and didn't roll off the tongue. Good nomenclature is important

When you're ideating in a space, you're really trying to map out the space, which is definitely a combinatorial problem with a fractal pattern. But the point is, you aren't trying to do a depth-first deductive process. You're coming up with a bunch of ideas for each axis and sub-axis. Once you do a lot of ideating, you have a lot done. You could really do several ideas based on just one ideation session.

One interesting pattern that I notice I run into is that whatever structure I'm using recursively retains the same fields when we go down a level. Use case example: I have panes for "villains, scary moments, core ideas, and locations". I want to be able to have scary moments that aren't attached to a villain (or perhaps that apply to more than one villain) but I also want each villain to have their own list of specific associated scary moments. Figuring out this kind of pattern is a necessary feature to launch.
- the solution to this is going to be a really sophisticated parent/child situation that allows for really easy modifying of the parent/child tree as well as really good defaults so that you don't have to laboriously set it up each time. Getting the right defaults and the right modification system will be really useful.
- handling this also means being able to "refresh" the boards and make another blank copy to go another direction. USE CASE: I was working on a scary story about the woods, when I decided to pivot to aliens. This creates a fork where I want to be able to take the structure (and some of the ideas) that I had going for my first story and then use them for my second story, without messing up the first story. This isn't a downstream for, it's actually a retroactive upstream fork. What's more, I want to be able to share a lot of the ideas (for example potential characters) between the two branches. But I don't want to be sharing the entire list of potential characters. In addition to excellent parent/child relationships, the relationships need to be granular (and have a good corresponding organization system) so that I can manage sharing only pieces between two ideation panes. So really, I need to do a great job managing not just parent/child, but the sharing web. The sharing web manages which ideas are shared with which other panes. In addition, you want to be able to organize and group the ideas vertically in the pane
- One limiting assumption that I have here is that the ideas need to be a simple vertical text box. A pane doesn't need to be a linear list of text. A great way to think about the pane is that it is just a view put on top of blocks in our web of ideas.

So now we are arriving at a whole new way to view the product. It isn't a tree (or even a web) of panes. We have several elements. I still need good nomenclature for this, but now I know we have the block (one item in a list of ideas) as the fundamental unit. We just have a sea of these, and we need to impose some sort of organization on them.  This 
